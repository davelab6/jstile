\Section{Execution}
\label{sec:execution}

  In the previous section, we provide an example to add a new kind of
tile called {\tt repeat} to the language.  However, how is the
semantics of it described?

  The macro for {\tt repeat} defines a new JavaScript object that is
used to represent the node in the parse tree.  For example, when the user
writes a code snippet with the macro as:
\begin{quote}
{\tt @repeat(10, alert("hello"))}
\end{quote}
, a {\tt repeat} object is created and its two fields are initialized
with these arguments (a parse tree for 10 and another parse tree for
{\tt alert("hello")}).

  As shown in Figure \ref{fig:macro repeat}, the macro specifies the
semantics, but the we would like to make it accessible to the
end-user.  Because the definition in Figure \ref{fig:macro repeat} is
in the same language, we can re-use the same GUI ``scriptor'' widget
for defining the expanded form.




%%   The semantics definition for this is done by writing a method called
%% {\tt eval()} in the same language.  For {\tt Repeat} case, the use
%% would write the {\tt eval()} method as follows.
%% \begin{figure}[th]
%% \begin{center}
%% \begin{verbatim}
%% Repeat.prototype.eval = function(ctxt) {
%%    var i = this.n(ctxt).eval(ctxt)
%%    while (i > 0) {
%%       i = i - 1
%%       body(ctxt).eval(ctxt)
%%    }
%% }
%% \end{verbatim}
%% \end{center}
%% \end{figure}

%% As you see, {\tt eval()} function recursively calls other elements'
%% {\tt eval()}.  The {\tt ctxt} argument is the dynamic execution
%% context so that if there is a variable reference in {\tt body}, it can
%% correctly look up the value.

%%   So far, it seems relatively simple. But our goal is not only provide
%% an easy way to define a new tile, but also ``isomorphic'' projection
%% of the tile language and textual language.  Each language element,
%% including {\tt while} used above, field access, and a method
%% definition have to have the tile representation.

%%   This means that for the {\tt while} syntax, we define {\tt While}
%% object and {\tt eval()} method, and for the {\tt function} syntax to
%% define a new function, we define {\tt Function} object and {\tt
%% eval()}.  In these {\tt eval()}s, we cannot use what we are
%% defining otherwise the meta-circularity doesn't terminate.

%% \begin{table}[th]
%% \caption{Primitives}
%% \label{tab:primitives}

%% \begin{tabular}{l|l}
%% \end{tabular}
%% \end{table}

%%   For this reason, we provide a 0-level construct for certain
%% primitive language construct.  Table \ref{tab:primitives} shows the
%% such primitives.  The {\tt eval()} for these primitives are also
%% written in JavaScript; but they are not the same thing.  (How?)

